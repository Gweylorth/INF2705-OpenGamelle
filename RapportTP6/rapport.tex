\documentclass[10pt,a4paper]{article}
\usepackage{geometry}
\usepackage[french]{babel}
\usepackage[utf8]{inputenc}
\usepackage[T1]{fontenc}
\usepackage{lmodern} \normalfont
\DeclareFontShape{T1}{lmr}{bx}{sc}{<-> ssub * cmr/bx/sc}{}
\usepackage{textcomp}
\usepackage{datetime}
\usepackage{amsmath}
\usepackage{amssymb}
\usepackage{graphicx}
\usepackage{wrapfig}
\usepackage{subcaption}
\usepackage{tocloft}
\usepackage{fixltx2e}
\usepackage{color}
\usepackage[colorlinks=true,
			linkcolor=blue,
			bookmarksnumbered=true,
			pdftitle={Rapport INF2705},
			pdfauthor={Pythagore Deffo, Gwenegan Hudin},
			pdfborder={0 0 0},
			pdfsubject={Rapport de laboratoire INF2705}]{hyperref}

% Custom commands
\newcommand{\HRule}{\rule{\linewidth}{0.5mm}}
\newcommand{\Section}[1]{\section*{#1} \addcontentsline{toc}{section}{#1} \setcounter{subsection}{0}}
%\renewcommand*{\theHsection}{chY.\the\value{section}}
\renewcommand{\thesection}{\Roman{section}.}
\renewcommand{\thesubsection}{\arabic{subsection}.}
\renewcommand{\thesubsubsection}{\alph{subsubsection}.}
\renewcommand{\cftsecnumwidth}{2em}
\renewcommand{\cftsubsecnumwidth}{2em}
\renewcommand{\cftsubsubsecnumwidth}{2em}
\addto\captionsfrench{
	\renewcommand{\cfttoctitlefont}{\Large}
	\renewcommand{\contentsname}{\centering \textsc{Table des Matières}\\[0.5cm]}
}

\renewcommand{\baselinestretch}{1.15}

\begin{document}

\begin{titlepage}
	\begin{center}
		\begin{figure}
        \begin{subfigure}[c]{0.2\textwidth}
        		\centering
                \includegraphics[width=0.6\textwidth]{images/logo-polymtl}
        \end{subfigure}
		\end{figure}
		
		
		\vspace{30pt}
		\textsc{\huge Génie Informatique}\\
		\textsc{\LARGE Rapport de Travaux Pratiques}\\		
		\vfill
		
		% Title
		\HRule \\[0.7cm]
		{\Huge \bfseries INF4705 - Infographie}\\[0.4cm]
		{\Large Lab 6 : Illumination et aplat de texture}\\[0.2cm]
		\HRule\\[1cm]
		
		\vfill
		
		% Author
		\begin{minipage}{0.49\textwidth}
			\begin{flushleft} \LARGE
				\textbf{Auteurs}\\
				Pythagore Raoul \textsc{Deffo}\\ 1635142\\
				Gwenegan \textsc{Hudin}\\ 1756642\\[0.5cm]
			\end{flushleft}
		\end{minipage}
		\begin{minipage}{0.49\textwidth}
			\begin{flushright} \LARGE
				\textbf{Rendu}\\
				11 Décembre 2014\\ À Polytechnique Montréal\\[0.5cm]
			\end{flushright}
		\end{minipage}
	\end{center}
\end{titlepage}

\newpage

\hfill

\newpage

\tableofcontents

\newpage

\section{Introduction}


\section{Exposé du problème}

\og Mentionnez la situation initiale, les objectifs des travaux pratiques et le contexte d'application du travail (le tout en vos mots). Remarque: un simple «copier/coller» des éléments demandés à partir des énoncés de laboratoire se verra attribuer la note «zéro». \fg

\section{Ajouts et modifications}

\og Décrivez les principaux ajouts ou modifications que vous avez faits ou que vous auriez pu faire, en rapport avec les exigences des TPs. \fg

\section{Discussion}

\og
    Discutez des points pertinents du code implémenté en rapport avec la théorie du cours. Au besoin, joindre le code pertinent en annexe.
    
    Dites si ce que vous avez réalisé est correct ou non, et en quoi son implantation respecte bien ou moins bien les requis.
    
    Listez les points forts et les points faibles de l'application et identifiez les causes.
    
    Mentionnez les améliorations et optimisations possibles et faisables.
\fg

\section{Difficultés}

\og 

    Identifiez les difficultés que vous avez rencontrées durant le développement (en spécifiant les fonctionnalités qui posaient des problèmes) et dites brièvement pourquoi c'était difficile.
    
    Discutez de ce que vous avez fait pour surmonter ces difficultés.

\fg

\section{Conclusion}

\section{Bibliographie}

\begin{itemize}
	\item \href{http://msdn.microsoft.com/en-us/library/296az74e.aspx}{Microsoft Developer Network, Integer Limits}
	\item \href{http://www.cplusplus.com/reference/algorithm/find/}{C++ Reference, Find algorithm}
	\item Notes de cours "Algorithmes Voraces", Gilles Pesant
	\item Notes de cours "Algorithmes Dynamiques", Gilles Pesant
\end{itemize}

\end{document}
