\documentclass[10pt,a4paper]{article}
\usepackage{geometry}
\usepackage[french]{babel}
\usepackage[utf8]{inputenc}
\usepackage[T1]{fontenc}
\usepackage{lmodern} \normalfont
\DeclareFontShape{T1}{lmr}{bx}{sc}{<-> ssub * cmr/bx/sc}{}
\usepackage{textcomp}
\usepackage{datetime}
\usepackage{amsmath}
\usepackage{amssymb}
\usepackage{graphicx}
\usepackage{wrapfig}
\usepackage{subcaption}
\usepackage{tocloft}
\usepackage{fixltx2e}
\usepackage{color}
\usepackage[colorlinks=true,
			linkcolor=blue,
			bookmarksnumbered=true,
			pdftitle={Rapport INF2705},
			pdfauthor={Pythagore Deffo, Gwenegan Hudin},
			pdfborder={0 0 0},
			pdfsubject={Rapport de laboratoire INF2705}]{hyperref}

% Custom commands
\newcommand{\HRule}{\rule{\linewidth}{0.5mm}}
\newcommand{\Section}[1]{\section*{#1} \addcontentsline{toc}{section}{#1} \setcounter{subsection}{0}}
%\renewcommand*{\theHsection}{chY.\the\value{section}}
\renewcommand{\thesection}{\Roman{section}.}
\renewcommand{\thesubsection}{\arabic{subsection}.}
\renewcommand{\thesubsubsection}{\alph{subsubsection}.}
\renewcommand{\cftsecnumwidth}{2em}
\renewcommand{\cftsubsecnumwidth}{2em}
\renewcommand{\cftsubsubsecnumwidth}{2em}
\addto\captionsfrench{
	\renewcommand{\cfttoctitlefont}{\Large}
	\renewcommand{\contentsname}{\centering \textsc{Table des Matières}\\[0.5cm]}
}

\renewcommand{\baselinestretch}{1.15}

\begin{document}

\begin{titlepage}
	\begin{center}
		\begin{figure}
        \begin{subfigure}[c]{0.2\textwidth}
        		\centering
                \includegraphics[width=0.6\textwidth]{images/logo-polymtl}
        \end{subfigure}
		\end{figure}
		
		
		\vspace{30pt}
		\textsc{\huge Génie Informatique}\\
		\textsc{\LARGE Rapport de Travaux Pratiques}\\		
		\vfill
		
		% Title
		\HRule \\[0.7cm]
		{\Huge \bfseries INF4705 - Infographie}\\[0.4cm]
		{\Large Lab 6 : Illumination et aplat de texture}\\[0.2cm]
		\HRule\\[1cm]
		
		\vfill
		
		% Author
		\begin{minipage}{0.49\textwidth}
			\begin{flushleft} \LARGE
				\textbf{Auteurs}\\
				Pythagore Raoul \textsc{Deffo}\\ 1635142\\
				Gwenegan \textsc{Hudin}\\ 1756642\\[0.5cm]
			\end{flushleft}
		\end{minipage}
		\begin{minipage}{0.49\textwidth}
			\begin{flushright} \LARGE
				\textbf{Rendu}\\
				11 Décembre 2014\\ À Polytechnique Montréal\\[0.5cm]
			\end{flushright}
		\end{minipage}
	\end{center}
\end{titlepage}

\newpage

\hfill

\newpage

\tableofcontents

\newpage

\section{Introduction}

Dans ce rapport, nous voulons revenir sur le sixième travail pratique du cours INF2707 - Infographie. Nous aborderons le problème initial, ce que nous avons ou aurions pu ajouter à la réalisation ; après quoi nous discuterons de points précis de notre implémentation, avant d'identifier les difficultés auxquelles nous avons été confrontés.

Pour réaliser ce TP, nous avions 6h de Lab dédiées.

\section{Exposé du problème}

Dans ce TP, nous devions mettre en application les méthodes d'illumination OpenGL évoquées en cours, ainsi que nous essayer à l'application de textures sur un objet 3D.

Les sources fournies au début de ce TP permettaient l'affichage d'une fenêtre OpenGL. Celle-ci présente un modèle 3D de cube, et une pression sur les touches numériques du clavier permet de changer pour un autre modèle (théière, sphère, tore, dodécahèdre, icosahèdre). Une source lumineuse, représentée par une petite sphère, peut être contrôlée avec la souris (position et orientation).

Nos objectifs sont multiples :

\begin{itemize}
	\item Utiliser une source lumineuse en spot, et appliquer les modèles d'illumination de Blinn et Phong par les nuanceurs. Les objets doivent aussi être éclairables sans nuanceurs.
	\item Modifier les paramètres du spot (angle maximal et exposant) par le clavier.
	\item Appliquer une illumination spot à la Direct3D, avec une atténuation sur les bords de la zone éclairée.
	\item Permettre l'illumination d'une des faces du cube sans nuanceur, par subdivision de celle-ci.
	\item Afficher deux textures (dé ou échiquier) sur le cube et la théière. Lorsqu'appliquée au cube, la texture du dé doit effectivement représenter un dé à 6 faces.
	\item Varier le rendu de la texture échiquier en modifiant le mode de répétition sur le cube.
	\item Rendre les pixels noirs transparents par les nuanceurs.
\end{itemize}

Comme pour les travaux précédents, nous avons réalisé ce TP en travaillant avec C++11, dans l'environnement de développement QT Creator 4 sous Fedora.

\section{Ajouts et modifications}

Nous avons respecté les exigences de ce TP sans les altérer.

Si nous avions eu plus de temps, nous aurions pu grandement améliorer notre mécanisme de subdivision de faces, en utilisant des \textit{VertexPointer}. Cela nous aurait permis de gagner en performances, et ainsi mieux subdiviser la face du cube. Cette subdivision pourrait aussi être étendue à toutes les faces du cube.

Cependant, le cube n'était pas le seul modèle à pâtir de l'éclairage OpenGL en pipeline fixe. Ainsi, la division de faces aurait pu être utilisée sur le dodécahèdre et l'icosahèdre. Ceux-ci sont néanmoins bien plus complexes à subdiviser, on ne peut simplement utiliser des \textit{Quads}. Nous aurions eu à nous pencher sur les problématiques de tesselation, et utiliser des triangles.

Enfin, comme nous avions un dé à 6 faces (un d6), il aurait été intéressant d'appliquer des textures similaires pour réaliser un d12 et un d20, respectivement sur le dodécahèdre et l'icosahèdre. Ainsi, nous aurions été parés pour une séance de Donjons \& Dragons !

\section{Discussion}

\og
    Discutez des points pertinents du code implémenté en rapport avec la théorie du cours. Au besoin, joindre le code pertinent en annexe.
    
    Dites si ce que vous avez réalisé est correct ou non, et en quoi son implantation respecte bien ou moins bien les requis.
    
    Listez les points forts et les points faibles de l'application et identifiez les causes.
    
    Mentionnez les améliorations et optimisations possibles et faisables.
\fg



\section{Difficultés}

\og 

    Identifiez les difficultés que vous avez rencontrées durant le développement (en spécifiant les fonctionnalités qui posaient des problèmes) et dites brièvement pourquoi c'était difficile.
    
    Discutez de ce que vous avez fait pour surmonter ces difficultés.

\fg

\section{Conclusion}

\section{Bibliographie}

\begin{itemize}
	\item \href{http://msdn.microsoft.com/en-us/library/296az74e.aspx}{Microsoft Developer Network, Integer Limits}
	\item \href{http://www.cplusplus.com/reference/algorithm/find/}{C++ Reference, Find algorithm}
	\item Notes de cours "Algorithmes Voraces", Gilles Pesant
	\item Notes de cours "Algorithmes Dynamiques", Gilles Pesant
\end{itemize}

\end{document}
